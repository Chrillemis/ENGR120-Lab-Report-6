We observe that the results from table \ref{table:results} of  $(5.3\pm 3.3)m^3/hr$ for Copper, $(4.3\pm 2.9)m^3/hr$ for galvanized steel and $(6.2\pm 1.0)m^3/hr$ for PVC are quite similar, which is a good sign. Although, the uncertainties are quite high. This is likely due to the large standard deviation of $a$, as can be seen in the figures \ref{fig:copper}, \ref{fig:galvanized} and \ref{fig:PVC}. To get this lower you could either obtain more data (run the experiment longer) as seen in eq. \ref{eq:deltaa}. Another, probably better way is to train a more complex model, eg. a neural network or a generalised linear model and use more data, both in the form of running the expirement for longer and for more flow rates, althoguh depending on the applications, simple linear regression (as in this report) may be adequate.