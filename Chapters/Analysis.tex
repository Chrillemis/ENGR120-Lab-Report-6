\begin{table}[h!]
\resizebox{\columnwidth}{!}{%
\begin{tabular}{lrrl}
\toprule
\textbf{Material} & \textbf{Axis} & \textbf{$R^2 score$} & \textbf{Unknown flowrate} \\
\midrule
copper & 0 & 0.73 & $5.3 \pm 3.3$ \\
copper & 1 & 0.38 & $6.0 \pm 1.8$ \\
copper & 2 & 0.50 & $0 \pm 8$ \\
galvanized & 0 & 0.64 & $4.3 \pm 2.9$ \\
galvanized & 1 & 0.27 & $3 \pm 4$ \\
galvanized & 2 & 0.39 & $10 \pm 9$ \\
pvc & 0 & 0.90 & $6.2 \pm 1.0$ \\
pvc & 1 & 0.07 & $-7 \pm 11$ \\
pvc & 2 & -0.37 & $1 \pm 5$ \\
\bottomrule
\end{tabular}

}
\caption{Table of $R^2$ score and calculated unknown flowrate for each material and axis}
\label{table:results}
\end{table}
As seen in Table \ref{table:results} it is clear, that $axis = 0$ is the best axis to apply our model from eq. \ref{eq:linfunc} on. This is done in figure \ref{fig:copper}, \ref{fig:galvanized} and \ref{fig:PVC}, from which the unknown flow rates are determined, as described in section \ref{sec_level1}. It is also clear from some of the flow rates, some having very large uncertainties and one even being negative.


\begin{figure}[h!]
    \centering
    \includegraphics[width=55mm]
    {Python/Lab6/Plots/copper_axis0.png}
    \caption{Copper} 
    \label{fig:copper}
\end{figure}

\begin{figure}[h!]
    \centering
    \includegraphics[width=55mm]
    {Python/Lab6/Plots/galvanized_axis0.png}
    \caption{Galvanized} 
    \label{fig:galvanized}
\end{figure}
  
\begin{figure}[h!]
    \centering
    \includegraphics[width=55mm]
    {Python/Lab6/Plots/pvc_axis0.png}
    \caption{PVC} 
    \label{fig:PVC}
\end{figure}

  





