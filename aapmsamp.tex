% ****** Start of file aapmsamp.tex ******
%
%   This file is part of the AAPM files in the AAPM distribution for REVTeX 4-2.
%   Version 4.2a of REVTeX, January 2015
%
%   Copyright (c) 2015 American Association of Physicists in Medicine (AAPM).
%
%   See the AAPM README file for restrictions and more information.
%
% TeX'ing this file requires that you have AMS-LaTeX 2.0 installed
% as well as the rest of the prerequisites for REVTeX 4.2
%
% It also requires running BibTeX. The commands are as follows:
%
%  1)  latex  aapmsamp
%  2)  bibtex aapmsamp
%  3)  latex  aapmsamp
%  4)  latex  aapmsamp
%
% Use this file as a source of example code for your aapm document.
% Use the file aapmtemplate.tex as a template for your document.
\documentclass[%
 aapm,
 mph,%
 amsmath,amssymb,
%preprint,%
 reprint,%
%author-year,%
%author-numerical,%
]{revtex4-2}
\usepackage{amssymb}
\usepackage{graphicx}
\usepackage{hyperref}
\usepackage{listings}
\usepackage{mathtools}
\usepackage{maple}
\usepackage[utf8]{inputenc}
\usepackage[svgnames]{xcolor}
\usepackage{amsmath}
\usepackage{breqn}
\usepackage{textcomp}
\usepackage{subcaption}


% \begin{document}
\lstset{basicstyle=\ttfamily,breaklines=true,columns=flexible}
\pagestyle{empty}
\DefineParaStyle{Maple Bullet Item}
\DefineParaStyle{Maple Heading 1}
\DefineParaStyle{Maple Warning}
\DefineParaStyle{Maple Heading 4}
\DefineParaStyle{Maple Heading 2}
\DefineParaStyle{Maple Heading 3}
\DefineParaStyle{Maple Dash Item}
\DefineParaStyle{Maple Error}
\DefineParaStyle{Maple Title}
\DefineParaStyle{Maple Ordered List 1}
\DefineParaStyle{Maple Text Output}
\DefineParaStyle{Maple Ordered List 2}
\DefineParaStyle{Maple Ordered List 3}
\DefineParaStyle{Maple Normal}
\DefineParaStyle{Maple Ordered List 4}
\DefineParaStyle{Maple Ordered List 5}
\DefineCharStyle{Maple 2D Output}
\DefineCharStyle{Maple 2D Input}
\DefineCharStyle{Maple Maple Input}
\DefineCharStyle{Maple 2D Math}
\DefineCharStyle{Maple Hyperlink}


\usepackage{gensymb}
\usepackage{graphicx}% Include figure files
\usepackage{dcolumn}% Align table columns on decimal point
\usepackage{physics}
% \usepackage{mathtools}
% \usepackage{amsmath}
\usepackage{bm}% bold math
\usepackage[mathlines]{lineno}% Enable numbering of text and display math
\usepackage{multirow}
\usepackage[normalem]{ulem}
\useunder{\uline}{\ul}{}
\modulolinenumbers[5]% Line numbers with a gap of 5 lines

\begin{document}
\noindent
\header{ENGR 120}
%\preprint{AAPM/123-QED}

\title[]{\underline{Fluid Mechanics: Lab 5} \\
Venturi and Orfice flow measurements and characterization}.% Force line breaks with \\

\author{Christian Emil Lorentsen}%
\author{Serafin Stauch}
\author{Bryan Saldivar}
\author{Xuan Zhao}

\affiliation{University of California, Merced}
\date{\today}% It is always \today, today,
             %  but any date may be explicitly specified
\begin{abstract}
In this report we take a look at the Ventury and Orifice meter to determine the volume flow rate through a pipe. This is done from two simple pressure measurements, at two different points in the pipe. We also apply the steady flow energy equation to our systems, to study the friction head loss in the meters, by measuring the pressure at three different points. Finally we also study the discharge coefficient in the Orifice meter.

\end{abstract}

\maketitle
\linenumbers\relax % Commence numbering lines
\begin{quotation}
%Explain why the physics you are studying is interesting and relevant.
Venturi and Orifice meters provides and excellent way of measuring volume flow rate. From two simple pressure reading it is possible to measure the volume flow rate in a pipe using a Venturi meter, making it useful in a large variety of application. An Orifice meter also makes us able to measure the flow rate through a pipe, this time only using a small hole.

\end{quotation}
\section{\label{sec:level1}Background}
First, we look at the Venturi meter. It consists of a circular pipe, with a given volume flow rate. At a section of the pipe, where it conically shrinks shortly, we measure the pressure at three different points, see fig \ref{fig:Venturi}. The Orifice meter is almost identical to the Venturi meter, although there is no conical shrinkring, but a \textit{wall} with a small hole in the middle of diameter $d$, see fig \ref{fig:Orifice}.

\subsection{\label{sec:level2}Theory}
From Bernoullis eq. we get the following for the Venturi meter:

\begin{eqnarray}
P_a +\frac{1}{2} V_a^2 + \rho g z_a = p_b + \frac{1}{2} \rho V_b ^2 + \rho g z_1 \\
\Rightarrow A_b^2 P_V = \frac{1}{2}A_b^2 \rho \big(V_b^2 - \tfrac{d_a^4}{d_b^4}V_b^2 \big) 
\end{eqnarray}
Where we get $V_a = (d_b^2/d_a^2)V_b$ from conservation of mass. This leads to the following equations:
\begin{eqnarray}
    Q = \frac{\pi}{4}d_b^2 \sqrt{\frac{2P_v}{\rho(1-d_a^4/d_b^4)}} \\
    P_v = \frac{8Q^2\rho(1-d_a^4/d_b^4)}{\pi^2d_b^2}
    \label{eq:Qventuri}
\end{eqnarray}

Eq. 373\cite{White_Xue_2021}
\begin{equation}
    \bigg(\frac{p}{\gamma} + \frac{V^2}{2g} + z\bigg)_{in} = \bigg(\frac{p}{\gamma} + \frac{V^2}{2g} + z\bigg)_{out} + h_{friction}
\end{equation}
Since we have no pump or turbine. $z_{in}=z_{out}$ and applying between point A and B we get
\begin{eqnarray}
    h_{f1} = \frac{P_v}{\rho g} + \frac{1}{2g}(V_a^2-V_b^2) = \frac{P_v}{\rho g} + \frac{Q^2}{2gA_b^2}\bigg(\frac{d_b^4}{d_a^4}-1\bigg) \\
    = \frac{P_v}{\rho g} + \frac{2Q^2}{g\pi^2d_b^4}\bigg(\frac{d_b^4}{d_a^4}-1\bigg)
    \label{eq:hf1}
\end{eqnarray}
It is easy to apply it between point A an C, since they have the same crossectional area, and we loose the second term:
\begin{eqnarray}
    h_{f2} = \frac{P_d}{\rho g}
    \label{eq:hf2}
\end{eqnarray}
For the Orifice plate the relationship between the flow rate and pressure drop can be derived as \cite{White_2021}
\begin{eqnarray}
    Q = C_d A_t \bigg[\frac{2\Delta p}{\rho(1-\beta^4)}\bigg]
    \label{eq:Cd}
\end{eqnarray}
Where $C_d$ is the discharge coefficient, $A_t$ is the throat area and $\beta = d/D$, see fig \ref{fig:Orifice}
\subsection{\label{sec_level3}Method \& Setup}



For both the Venturi and Orifice meter we choose 6 values of $Q$ and measure $P_d$, $P_v$ and $\Delta p$ as seen in fig \ref{fig:setup}

\begin{figure}[h!]
  \begin{subfigure}{0.51\columnwidth}
    \includegraphics[width=\linewidth]{Diagrams/Venturi.jpeg}
    \caption{Venturi meter} 
    \label{fig:Venturi}
  \end{subfigure}%
  \hspace*{\fill}   % maximize separation between the subfigures
  \begin{subfigure}{0.435\columnwidth}
    \includegraphics[width=\linewidth]{Diagrams/Orifice.jpeg}
    \caption{Orifice meter} 
    \label{fig:Orifice}
  \end{subfigure}%

\label{fig:setup}
\caption{Overview of the two different systems in use} \label{fig:Meters}
\end{figure}

\newpage
\section{DATA AND RESULTS}
% \textbf{Fra Jasons slides:} \\
% \textit{
% Include the data you took as figures or tables. \\
% Any figures and tables should have appropriate captions and must be referenced somewhere in the main text. \\
% Be sure that the plots are properly formatted with axes labels, units, and any relevant error bars on data. 
% }

Listed in Table \ref{table:Data} are the values measured from the setups described in \ref{sec_level3}. Also there are calculated values, where the Ideal $P_v$ is calculated using eq. \ref{eq:Qventuri} and $C_d$ from eq. \ref{eq:Cd}

% Please add the following required packages to your document preamble:
% \usepackage{graphicx}
\begin{table}[h!]
\resizebox{\columnwidth}{!}{%
\begin{tabular}{l|lll|ll|}
\cline{2-6}
                                           & \multicolumn{3}{l|}{Venturi Meter}                                                                              & \multicolumn{2}{l|}{Orifice meter}                                        \\ \hline
\multicolumn{1}{|l|}{\textbf{Q $[m^3/s]$}} & \multicolumn{1}{l|}{Measured $P_v [Pa]$} & \multicolumn{1}{l|}{\textbf{Measured $P_d [Pa]$}} & Ideal $P_v [Pa]$ & \multicolumn{1}{l|}{\textbf{$\delta p [Pa]$}} & \textbf{Calculated $C_d [m^{-1}]$} \\ \hline
\multicolumn{1}{|l|}{$16.67*10^{-6}$}         & \multicolumn{1}{l|}{$-5350$}             & \multicolumn{1}{l|}{$-3450$}                      & $-241.8$         & \multicolumn{1}{l|}{$1850.00$}                & $0.63$                    \\ \hline
\multicolumn{1}{|l|}{$25.00*10^{-6}$}         & \multicolumn{1}{l|}{$-9950$}             & \multicolumn{1}{l|}{$-5950$}                      & $-544.0$         & \multicolumn{1}{l|}{$4050.00$}                & $0.64$                    \\ \hline
\multicolumn{1}{|l|}{$33.33*10^{-6}$}         & \multicolumn{1}{l|}{$-17400$}            & \multicolumn{1}{l|}{$-8850$}                      & $-967.2$         & \multicolumn{1}{l|}{$7200.00$}                & $0.64$                    \\ \hline
\multicolumn{1}{|l|}{$41.67*10^{-6}$}         & \multicolumn{1}{l|}{$-27200$}            & \multicolumn{1}{l|}{$-12250$}                     & $-1511.2$        & \multicolumn{1}{l|}{$11450.00$}               & $0.64$                    \\ \hline
\multicolumn{1}{|l|}{$50.00*10^{-6}$}         & \multicolumn{1}{l|}{$-38200$}            & \multicolumn{1}{l|}{$-16300$}                     & $-2176.2$        & \multicolumn{1}{l|}{$16550.00$}               & $0.64$                    \\ \hline
\multicolumn{1}{|l|}{$58.33*10^{-6}$}         & \multicolumn{1}{l|}{$-49650$}            & \multicolumn{1}{l|}{$-20650$}                     & $-2962.0$        & \multicolumn{1}{l|}{$23350.00$}               & $0.62$                    \\ \hline
\end{tabular}%
}
\caption{Data for Venturi meter with an average value for $C_d\approx0.636 $.}

\label{table:Data}
\end{table}



% Please add the following required packages to your document preamble:
% \usepackage{graphicx}

\subsection{Analysis}



\begin{figure}[h!]
    \centering
    \includegraphics[width=55mm]
    {Python/Lab6/Plots/copper_axis0.png}
    \caption{Copper} 
    \label{fig:copper}
\end{figure}

\begin{figure}[h!]
    \centering
    \includegraphics[width=55mm]
    {Python/Lab6/Plots/galvanized_axis0.png}
    \caption{Galvanized} 
    \label{fig:galvanized}
\end{figure}
  
\begin{figure}[h!]
    \centering
    \includegraphics[width=55mm]
    {Python/Lab6/Plots/pvc_axis0.png}
    \caption{PVC} 
    \label{fig:PVC}
\end{figure}

  




\begin{table}[h!]
\resizebox{\columnwidth}{!}{%
\begin{tabular}{lrrl}
\toprule
\textbf{Material} & \textbf{Axis} & \textbf{$R^2 score$} & \textbf{Unknown flowrate} \\
\midrule
copper & 0 & 0.73 & $5.3 \pm 3.3$ \\
copper & 1 & 0.38 & $6.0 \pm 1.8$ \\
copper & 2 & 0.50 & $0 \pm 8$ \\
galvanized & 0 & 0.64 & $4.3 \pm 2.9$ \\
galvanized & 1 & 0.27 & $3 \pm 4$ \\
galvanized & 2 & 0.39 & $10 \pm 9$ \\
pvc & 0 & 0.90 & $6.2 \pm 1.0$ \\
pvc & 1 & 0.07 & $-7 \pm 11$ \\
pvc & 2 & -0.37 & $1 \pm 5$ \\
\bottomrule
\end{tabular}

}
\caption{Table of calculated  \ref{eq:hf1} and \ref{eq:hf2}}
\label{table:hf1hf2}


\end{table}


% \textbf{Fra Jasons slides:} \\
% \textit{
% Include any systematic and statistical uncertainties. Explicitly write, describe, and quantify the impact of systematic uncertainties as part of the main text. \\
% Report the outcome of any calculations or fits using the data. All results and calculations should include uncertainties.
% }
\newpage
\section{DISCUSSION}
We observe that the data deviates a lot from the ideal values in Figure \ref{fig:pvVsQ}. This is likely due to the fact that we do not account for friction and viscous forces.

From \ref{fig:hf2VsQ} it is very clear that $h_{f2}$ is quadratically dependent on $Q$, which is why we in figure \ref{fig:hf1Vshf2} observe that $h_{f1}$ and $h_{f2}$ are very correlated, almost following a straight line. This is due to the fact that they are both functions of $Q^2$.



\section{CONCLUSION}
In conclusion we have succesfully estimated the discharge coefficient, $C_d$ for a Orifice meter at $C_d \approx 0.363m^{-1}$. We have also concluded that the friction heads in a venturi meter $h_{f1}$ and $h_{f2}$ are highly correlated and quadratically dependent on $Q$. Our measurements for $P_d$ are unfortunately too low, although we haven't taken friction, viscous forces and turbulent into account

% \newpage


%\textbf{Fra Jasons slides:} \\
%\textit{
%Discuss the results! Did your measurement and calculation agree with theory? Were %your systematic uncertainties possibly too few or too many? Is there something you %%would change if you redo the experiment? What is the most important outcome and the %thing that you want the reader to remember most vividly?
%}
{
\appendix
% \section{Navn på appendix}
% \textbf{Fra Jasons slides:} \\
% \textit{
% Marginally relevant material, equations, or derivations that support the write-up, but are not strictly necessary to understand the results, can be put in an appendix. \\
% Lots of work done for the prelabs could be included in an appendix. E.g., while the equations themselves should absolutely be included the main text, the derivations of the magnetic field strength or the Schrödinger equation can be included in an appendix.\\
% Appendices are not required.
% }
% \section{\label{app:schr}The time dependent Schrödinger equation}
% The equation takes the form


% \section{\label{app:prelab1}}



\section{Calculations}
% \subsection{\label{app:maple}Theoretical value of $C_d$}
% The following was run in Maple
% 
% \begin{Maple Normal}
% {$ \displaystyle \int_{0}^{\textcolor{DarkOrchid}{\pi}}(1-4\cdot \sin^{2}(\theta))\cdot \cos (\theta)d \theta  $}
% \end{Maple Normal}
% % \mapleresult
% \begin{dmath}\label{(1)}
% 0
% \end{dmath}
% \begin{Maple Normal}

% \end{Maple Normal}
% \begin{Maple Normal}

% \end{Maple Normal}

\subsection{\label{app:python}Python code used to run the calculations and the numerical integration}
\documentclass[11pt]{article}

    \usepackage[breakable]{tcolorbox}
    \usepackage{parskip} % Stop auto-indenting (to mimic markdown behaviour)
    

    % Basic figure setup, for now with no caption control since it's done
    % automatically by Pandoc (which extracts ![](path) syntax from Markdown).
    \usepackage{graphicx}
    % Keep aspect ratio if custom image width or height is specified
    \setkeys{Gin}{keepaspectratio}
    % Maintain compatibility with old templates. Remove in nbconvert 6.0
    \let\Oldincludegraphics\includegraphics
    % Ensure that by default, figures have no caption (until we provide a
    % proper Figure object with a Caption API and a way to capture that
    % in the conversion process - todo).
    \usepackage{caption}
    \DeclareCaptionFormat{nocaption}{}
    \captionsetup{format=nocaption,aboveskip=0pt,belowskip=0pt}

    \usepackage{float}
    \floatplacement{figure}{H} % forces figures to be placed at the correct location
    \usepackage{xcolor} % Allow colors to be defined
    \usepackage{enumerate} % Needed for markdown enumerations to work
    \usepackage{geometry} % Used to adjust the document margins
    \usepackage{amsmath} % Equations
    \usepackage{amssymb} % Equations
    \usepackage{textcomp} % defines textquotesingle
    % Hack from http://tex.stackexchange.com/a/47451/13684:
    \AtBeginDocument{%
        \def\PYZsq{\textquotesingle}% Upright quotes in Pygmentized code
    }
    \usepackage{upquote} % Upright quotes for verbatim code
    \usepackage{eurosym} % defines \euro

    \usepackage{iftex}
    \ifPDFTeX
        \usepackage[T1]{fontenc}
        \IfFileExists{alphabeta.sty}{
              \usepackage{alphabeta}
          }{
              \usepackage[mathletters]{ucs}
              \usepackage[utf8x]{inputenc}
          }
    \else
        \usepackage{fontspec}
        \usepackage{unicode-math}
    \fi

    \usepackage{fancyvrb} % verbatim replacement that allows latex
    \usepackage{grffile} % extends the file name processing of package graphics
                         % to support a larger range
    \makeatletter % fix for old versions of grffile with XeLaTeX
    \@ifpackagelater{grffile}{2019/11/01}
    {
      % Do nothing on new versions
    }
    {
      \def\Gread@@xetex#1{%
        \IfFileExists{"\Gin@base".bb}%
        {\Gread@eps{\Gin@base.bb}}%
        {\Gread@@xetex@aux#1}%
      }
    }
    \makeatother
    \usepackage[Export]{adjustbox} % Used to constrain images to a maximum size
    \adjustboxset{max size={0.9\linewidth}{0.9\paperheight}}

    % The hyperref package gives us a pdf with properly built
    % internal navigation ('pdf bookmarks' for the table of contents,
    % internal cross-reference links, web links for URLs, etc.)
    \usepackage{hyperref}
    % The default LaTeX title has an obnoxious amount of whitespace. By default,
    % titling removes some of it. It also provides customization options.
    \usepackage{titling}
    \usepackage{longtable} % longtable support required by pandoc >1.10
    \usepackage{booktabs}  % table support for pandoc > 1.12.2
    \usepackage{array}     % table support for pandoc >= 2.11.3
    \usepackage{calc}      % table minipage width calculation for pandoc >= 2.11.1
    \usepackage[inline]{enumitem} % IRkernel/repr support (it uses the enumerate* environment)
    \usepackage[normalem]{ulem} % ulem is needed to support strikethroughs (\sout)
                                % normalem makes italics be italics, not underlines
    \usepackage{soul}      % strikethrough (\st) support for pandoc >= 3.0.0
    \usepackage{mathrsfs}
    

    
    % Colors for the hyperref package
    \definecolor{urlcolor}{rgb}{0,.145,.698}
    \definecolor{linkcolor}{rgb}{.71,0.21,0.01}
    \definecolor{citecolor}{rgb}{.12,.54,.11}

    % ANSI colors
    \definecolor{ansi-black}{HTML}{3E424D}
    \definecolor{ansi-black-intense}{HTML}{282C36}
    \definecolor{ansi-red}{HTML}{E75C58}
    \definecolor{ansi-red-intense}{HTML}{B22B31}
    \definecolor{ansi-green}{HTML}{00A250}
    \definecolor{ansi-green-intense}{HTML}{007427}
    \definecolor{ansi-yellow}{HTML}{DDB62B}
    \definecolor{ansi-yellow-intense}{HTML}{B27D12}
    \definecolor{ansi-blue}{HTML}{208FFB}
    \definecolor{ansi-blue-intense}{HTML}{0065CA}
    \definecolor{ansi-magenta}{HTML}{D160C4}
    \definecolor{ansi-magenta-intense}{HTML}{A03196}
    \definecolor{ansi-cyan}{HTML}{60C6C8}
    \definecolor{ansi-cyan-intense}{HTML}{258F8F}
    \definecolor{ansi-white}{HTML}{C5C1B4}
    \definecolor{ansi-white-intense}{HTML}{A1A6B2}
    \definecolor{ansi-default-inverse-fg}{HTML}{FFFFFF}
    \definecolor{ansi-default-inverse-bg}{HTML}{000000}

    % common color for the border for error outputs.
    \definecolor{outerrorbackground}{HTML}{FFDFDF}

    % commands and environments needed by pandoc snippets
    % extracted from the output of `pandoc -s`
    \providecommand{\tightlist}{%
      \setlength{\itemsep}{0pt}\setlength{\parskip}{0pt}}
    \DefineVerbatimEnvironment{Highlighting}{Verbatim}{commandchars=\\\{\}}
    % Add ',fontsize=\small' for more characters per line
    \newenvironment{Shaded}{}{}
    \newcommand{\KeywordTok}[1]{\textcolor[rgb]{0.00,0.44,0.13}{\textbf{{#1}}}}
    \newcommand{\DataTypeTok}[1]{\textcolor[rgb]{0.56,0.13,0.00}{{#1}}}
    \newcommand{\DecValTok}[1]{\textcolor[rgb]{0.25,0.63,0.44}{{#1}}}
    \newcommand{\BaseNTok}[1]{\textcolor[rgb]{0.25,0.63,0.44}{{#1}}}
    \newcommand{\FloatTok}[1]{\textcolor[rgb]{0.25,0.63,0.44}{{#1}}}
    \newcommand{\CharTok}[1]{\textcolor[rgb]{0.25,0.44,0.63}{{#1}}}
    \newcommand{\StringTok}[1]{\textcolor[rgb]{0.25,0.44,0.63}{{#1}}}
    \newcommand{\CommentTok}[1]{\textcolor[rgb]{0.38,0.63,0.69}{\textit{{#1}}}}
    \newcommand{\OtherTok}[1]{\textcolor[rgb]{0.00,0.44,0.13}{{#1}}}
    \newcommand{\AlertTok}[1]{\textcolor[rgb]{1.00,0.00,0.00}{\textbf{{#1}}}}
    \newcommand{\FunctionTok}[1]{\textcolor[rgb]{0.02,0.16,0.49}{{#1}}}
    \newcommand{\RegionMarkerTok}[1]{{#1}}
    \newcommand{\ErrorTok}[1]{\textcolor[rgb]{1.00,0.00,0.00}{\textbf{{#1}}}}
    \newcommand{\NormalTok}[1]{{#1}}

    % Additional commands for more recent versions of Pandoc
    \newcommand{\ConstantTok}[1]{\textcolor[rgb]{0.53,0.00,0.00}{{#1}}}
    \newcommand{\SpecialCharTok}[1]{\textcolor[rgb]{0.25,0.44,0.63}{{#1}}}
    \newcommand{\VerbatimStringTok}[1]{\textcolor[rgb]{0.25,0.44,0.63}{{#1}}}
    \newcommand{\SpecialStringTok}[1]{\textcolor[rgb]{0.73,0.40,0.53}{{#1}}}
    \newcommand{\ImportTok}[1]{{#1}}
    \newcommand{\DocumentationTok}[1]{\textcolor[rgb]{0.73,0.13,0.13}{\textit{{#1}}}}
    \newcommand{\AnnotationTok}[1]{\textcolor[rgb]{0.38,0.63,0.69}{\textbf{\textit{{#1}}}}}
    \newcommand{\CommentVarTok}[1]{\textcolor[rgb]{0.38,0.63,0.69}{\textbf{\textit{{#1}}}}}
    \newcommand{\VariableTok}[1]{\textcolor[rgb]{0.10,0.09,0.49}{{#1}}}
    \newcommand{\ControlFlowTok}[1]{\textcolor[rgb]{0.00,0.44,0.13}{\textbf{{#1}}}}
    \newcommand{\OperatorTok}[1]{\textcolor[rgb]{0.40,0.40,0.40}{{#1}}}
    \newcommand{\BuiltInTok}[1]{{#1}}
    \newcommand{\ExtensionTok}[1]{{#1}}
    \newcommand{\PreprocessorTok}[1]{\textcolor[rgb]{0.74,0.48,0.00}{{#1}}}
    \newcommand{\AttributeTok}[1]{\textcolor[rgb]{0.49,0.56,0.16}{{#1}}}
    \newcommand{\InformationTok}[1]{\textcolor[rgb]{0.38,0.63,0.69}{\textbf{\textit{{#1}}}}}
    \newcommand{\WarningTok}[1]{\textcolor[rgb]{0.38,0.63,0.69}{\textbf{\textit{{#1}}}}}


    % Define a nice break command that doesn't care if a line doesn't already
    % exist.
    \def\br{\hspace*{\fill} \\* }
    % Math Jax compatibility definitions
    \def\gt{>}
    \def\lt{<}
    \let\Oldtex\TeX
    \let\Oldlatex\LaTeX
    \renewcommand{\TeX}{\textrm{\Oldtex}}
    \renewcommand{\LaTeX}{\textrm{\Oldlatex}}
    % Document parameters
    % Document title
    \title{Lab6}
    
    
    
    
    
    
    
% Pygments definitions
\makeatletter
\def\PY@reset{\let\PY@it=\relax \let\PY@bf=\relax%
    \let\PY@ul=\relax \let\PY@tc=\relax%
    \let\PY@bc=\relax \let\PY@ff=\relax}
\def\PY@tok#1{\csname PY@tok@#1\endcsname}
\def\PY@toks#1+{\ifx\relax#1\empty\else%
    \PY@tok{#1}\expandafter\PY@toks\fi}
\def\PY@do#1{\PY@bc{\PY@tc{\PY@ul{%
    \PY@it{\PY@bf{\PY@ff{#1}}}}}}}
\def\PY#1#2{\PY@reset\PY@toks#1+\relax+\PY@do{#2}}

\@namedef{PY@tok@w}{\def\PY@tc##1{\textcolor[rgb]{0.73,0.73,0.73}{##1}}}
\@namedef{PY@tok@c}{\let\PY@it=\textit\def\PY@tc##1{\textcolor[rgb]{0.24,0.48,0.48}{##1}}}
\@namedef{PY@tok@cp}{\def\PY@tc##1{\textcolor[rgb]{0.61,0.40,0.00}{##1}}}
\@namedef{PY@tok@k}{\let\PY@bf=\textbf\def\PY@tc##1{\textcolor[rgb]{0.00,0.50,0.00}{##1}}}
\@namedef{PY@tok@kp}{\def\PY@tc##1{\textcolor[rgb]{0.00,0.50,0.00}{##1}}}
\@namedef{PY@tok@kt}{\def\PY@tc##1{\textcolor[rgb]{0.69,0.00,0.25}{##1}}}
\@namedef{PY@tok@o}{\def\PY@tc##1{\textcolor[rgb]{0.40,0.40,0.40}{##1}}}
\@namedef{PY@tok@ow}{\let\PY@bf=\textbf\def\PY@tc##1{\textcolor[rgb]{0.67,0.13,1.00}{##1}}}
\@namedef{PY@tok@nb}{\def\PY@tc##1{\textcolor[rgb]{0.00,0.50,0.00}{##1}}}
\@namedef{PY@tok@nf}{\def\PY@tc##1{\textcolor[rgb]{0.00,0.00,1.00}{##1}}}
\@namedef{PY@tok@nc}{\let\PY@bf=\textbf\def\PY@tc##1{\textcolor[rgb]{0.00,0.00,1.00}{##1}}}
\@namedef{PY@tok@nn}{\let\PY@bf=\textbf\def\PY@tc##1{\textcolor[rgb]{0.00,0.00,1.00}{##1}}}
\@namedef{PY@tok@ne}{\let\PY@bf=\textbf\def\PY@tc##1{\textcolor[rgb]{0.80,0.25,0.22}{##1}}}
\@namedef{PY@tok@nv}{\def\PY@tc##1{\textcolor[rgb]{0.10,0.09,0.49}{##1}}}
\@namedef{PY@tok@no}{\def\PY@tc##1{\textcolor[rgb]{0.53,0.00,0.00}{##1}}}
\@namedef{PY@tok@nl}{\def\PY@tc##1{\textcolor[rgb]{0.46,0.46,0.00}{##1}}}
\@namedef{PY@tok@ni}{\let\PY@bf=\textbf\def\PY@tc##1{\textcolor[rgb]{0.44,0.44,0.44}{##1}}}
\@namedef{PY@tok@na}{\def\PY@tc##1{\textcolor[rgb]{0.41,0.47,0.13}{##1}}}
\@namedef{PY@tok@nt}{\let\PY@bf=\textbf\def\PY@tc##1{\textcolor[rgb]{0.00,0.50,0.00}{##1}}}
\@namedef{PY@tok@nd}{\def\PY@tc##1{\textcolor[rgb]{0.67,0.13,1.00}{##1}}}
\@namedef{PY@tok@s}{\def\PY@tc##1{\textcolor[rgb]{0.73,0.13,0.13}{##1}}}
\@namedef{PY@tok@sd}{\let\PY@it=\textit\def\PY@tc##1{\textcolor[rgb]{0.73,0.13,0.13}{##1}}}
\@namedef{PY@tok@si}{\let\PY@bf=\textbf\def\PY@tc##1{\textcolor[rgb]{0.64,0.35,0.47}{##1}}}
\@namedef{PY@tok@se}{\let\PY@bf=\textbf\def\PY@tc##1{\textcolor[rgb]{0.67,0.36,0.12}{##1}}}
\@namedef{PY@tok@sr}{\def\PY@tc##1{\textcolor[rgb]{0.64,0.35,0.47}{##1}}}
\@namedef{PY@tok@ss}{\def\PY@tc##1{\textcolor[rgb]{0.10,0.09,0.49}{##1}}}
\@namedef{PY@tok@sx}{\def\PY@tc##1{\textcolor[rgb]{0.00,0.50,0.00}{##1}}}
\@namedef{PY@tok@m}{\def\PY@tc##1{\textcolor[rgb]{0.40,0.40,0.40}{##1}}}
\@namedef{PY@tok@gh}{\let\PY@bf=\textbf\def\PY@tc##1{\textcolor[rgb]{0.00,0.00,0.50}{##1}}}
\@namedef{PY@tok@gu}{\let\PY@bf=\textbf\def\PY@tc##1{\textcolor[rgb]{0.50,0.00,0.50}{##1}}}
\@namedef{PY@tok@gd}{\def\PY@tc##1{\textcolor[rgb]{0.63,0.00,0.00}{##1}}}
\@namedef{PY@tok@gi}{\def\PY@tc##1{\textcolor[rgb]{0.00,0.52,0.00}{##1}}}
\@namedef{PY@tok@gr}{\def\PY@tc##1{\textcolor[rgb]{0.89,0.00,0.00}{##1}}}
\@namedef{PY@tok@ge}{\let\PY@it=\textit}
\@namedef{PY@tok@gs}{\let\PY@bf=\textbf}
\@namedef{PY@tok@gp}{\let\PY@bf=\textbf\def\PY@tc##1{\textcolor[rgb]{0.00,0.00,0.50}{##1}}}
\@namedef{PY@tok@go}{\def\PY@tc##1{\textcolor[rgb]{0.44,0.44,0.44}{##1}}}
\@namedef{PY@tok@gt}{\def\PY@tc##1{\textcolor[rgb]{0.00,0.27,0.87}{##1}}}
\@namedef{PY@tok@err}{\def\PY@bc##1{{\setlength{\fboxsep}{\string -\fboxrule}\fcolorbox[rgb]{1.00,0.00,0.00}{1,1,1}{\strut ##1}}}}
\@namedef{PY@tok@kc}{\let\PY@bf=\textbf\def\PY@tc##1{\textcolor[rgb]{0.00,0.50,0.00}{##1}}}
\@namedef{PY@tok@kd}{\let\PY@bf=\textbf\def\PY@tc##1{\textcolor[rgb]{0.00,0.50,0.00}{##1}}}
\@namedef{PY@tok@kn}{\let\PY@bf=\textbf\def\PY@tc##1{\textcolor[rgb]{0.00,0.50,0.00}{##1}}}
\@namedef{PY@tok@kr}{\let\PY@bf=\textbf\def\PY@tc##1{\textcolor[rgb]{0.00,0.50,0.00}{##1}}}
\@namedef{PY@tok@bp}{\def\PY@tc##1{\textcolor[rgb]{0.00,0.50,0.00}{##1}}}
\@namedef{PY@tok@fm}{\def\PY@tc##1{\textcolor[rgb]{0.00,0.00,1.00}{##1}}}
\@namedef{PY@tok@vc}{\def\PY@tc##1{\textcolor[rgb]{0.10,0.09,0.49}{##1}}}
\@namedef{PY@tok@vg}{\def\PY@tc##1{\textcolor[rgb]{0.10,0.09,0.49}{##1}}}
\@namedef{PY@tok@vi}{\def\PY@tc##1{\textcolor[rgb]{0.10,0.09,0.49}{##1}}}
\@namedef{PY@tok@vm}{\def\PY@tc##1{\textcolor[rgb]{0.10,0.09,0.49}{##1}}}
\@namedef{PY@tok@sa}{\def\PY@tc##1{\textcolor[rgb]{0.73,0.13,0.13}{##1}}}
\@namedef{PY@tok@sb}{\def\PY@tc##1{\textcolor[rgb]{0.73,0.13,0.13}{##1}}}
\@namedef{PY@tok@sc}{\def\PY@tc##1{\textcolor[rgb]{0.73,0.13,0.13}{##1}}}
\@namedef{PY@tok@dl}{\def\PY@tc##1{\textcolor[rgb]{0.73,0.13,0.13}{##1}}}
\@namedef{PY@tok@s2}{\def\PY@tc##1{\textcolor[rgb]{0.73,0.13,0.13}{##1}}}
\@namedef{PY@tok@sh}{\def\PY@tc##1{\textcolor[rgb]{0.73,0.13,0.13}{##1}}}
\@namedef{PY@tok@s1}{\def\PY@tc##1{\textcolor[rgb]{0.73,0.13,0.13}{##1}}}
\@namedef{PY@tok@mb}{\def\PY@tc##1{\textcolor[rgb]{0.40,0.40,0.40}{##1}}}
\@namedef{PY@tok@mf}{\def\PY@tc##1{\textcolor[rgb]{0.40,0.40,0.40}{##1}}}
\@namedef{PY@tok@mh}{\def\PY@tc##1{\textcolor[rgb]{0.40,0.40,0.40}{##1}}}
\@namedef{PY@tok@mi}{\def\PY@tc##1{\textcolor[rgb]{0.40,0.40,0.40}{##1}}}
\@namedef{PY@tok@il}{\def\PY@tc##1{\textcolor[rgb]{0.40,0.40,0.40}{##1}}}
\@namedef{PY@tok@mo}{\def\PY@tc##1{\textcolor[rgb]{0.40,0.40,0.40}{##1}}}
\@namedef{PY@tok@ch}{\let\PY@it=\textit\def\PY@tc##1{\textcolor[rgb]{0.24,0.48,0.48}{##1}}}
\@namedef{PY@tok@cm}{\let\PY@it=\textit\def\PY@tc##1{\textcolor[rgb]{0.24,0.48,0.48}{##1}}}
\@namedef{PY@tok@cpf}{\let\PY@it=\textit\def\PY@tc##1{\textcolor[rgb]{0.24,0.48,0.48}{##1}}}
\@namedef{PY@tok@c1}{\let\PY@it=\textit\def\PY@tc##1{\textcolor[rgb]{0.24,0.48,0.48}{##1}}}
\@namedef{PY@tok@cs}{\let\PY@it=\textit\def\PY@tc##1{\textcolor[rgb]{0.24,0.48,0.48}{##1}}}

\def\PYZbs{\char`\\}
\def\PYZus{\char`\_}
\def\PYZob{\char`\{}
\def\PYZcb{\char`\}}
\def\PYZca{\char`\^}
\def\PYZam{\char`\&}
\def\PYZlt{\char`\<}
\def\PYZgt{\char`\>}
\def\PYZsh{\char`\#}
\def\PYZpc{\char`\%}
\def\PYZdl{\char`\$}
\def\PYZhy{\char`\-}
\def\PYZsq{\char`\'}
\def\PYZdq{\char`\"}
\def\PYZti{\char`\~}
% for compatibility with earlier versions
\def\PYZat{@}
\def\PYZlb{[}
\def\PYZrb{]}
\makeatother


    % For linebreaks inside Verbatim environment from package fancyvrb.
    \makeatletter
        \newbox\Wrappedcontinuationbox
        \newbox\Wrappedvisiblespacebox
        \newcommand*\Wrappedvisiblespace {\textcolor{red}{\textvisiblespace}}
        \newcommand*\Wrappedcontinuationsymbol {\textcolor{red}{\llap{\tiny$\m@th\hookrightarrow$}}}
        \newcommand*\Wrappedcontinuationindent {3ex }
        \newcommand*\Wrappedafterbreak {\kern\Wrappedcontinuationindent\copy\Wrappedcontinuationbox}
        % Take advantage of the already applied Pygments mark-up to insert
        % potential linebreaks for TeX processing.
        %        {, <, #, %, $, ' and ": go to next line.
        %        _, }, ^, &, >, - and ~: stay at end of broken line.
        % Use of \textquotesingle for straight quote.
        \newcommand*\Wrappedbreaksatspecials {%
            \def\PYGZus{\discretionary{\char`\_}{\Wrappedafterbreak}{\char`\_}}%
            \def\PYGZob{\discretionary{}{\Wrappedafterbreak\char`\{}{\char`\{}}%
            \def\PYGZcb{\discretionary{\char`\}}{\Wrappedafterbreak}{\char`\}}}%
            \def\PYGZca{\discretionary{\char`\^}{\Wrappedafterbreak}{\char`\^}}%
            \def\PYGZam{\discretionary{\char`\&}{\Wrappedafterbreak}{\char`\&}}%
            \def\PYGZlt{\discretionary{}{\Wrappedafterbreak\char`\<}{\char`\<}}%
            \def\PYGZgt{\discretionary{\char`\>}{\Wrappedafterbreak}{\char`\>}}%
            \def\PYGZsh{\discretionary{}{\Wrappedafterbreak\char`\#}{\char`\#}}%
            \def\PYGZpc{\discretionary{}{\Wrappedafterbreak\char`\%}{\char`\%}}%
            \def\PYGZdl{\discretionary{}{\Wrappedafterbreak\char`\$}{\char`\$}}%
            \def\PYGZhy{\discretionary{\char`\-}{\Wrappedafterbreak}{\char`\-}}%
            \def\PYGZsq{\discretionary{}{\Wrappedafterbreak\textquotesingle}{\textquotesingle}}%
            \def\PYGZdq{\discretionary{}{\Wrappedafterbreak\char`\"}{\char`\"}}%
            \def\PYGZti{\discretionary{\char`\~}{\Wrappedafterbreak}{\char`\~}}%
        }
        % Some characters . , ; ? ! / are not pygmentized.
        % This macro makes them "active" and they will insert potential linebreaks
        \newcommand*\Wrappedbreaksatpunct {%
            \lccode`\~`\.\lowercase{\def~}{\discretionary{\hbox{\char`\.}}{\Wrappedafterbreak}{\hbox{\char`\.}}}%
            \lccode`\~`\,\lowercase{\def~}{\discretionary{\hbox{\char`\,}}{\Wrappedafterbreak}{\hbox{\char`\,}}}%
            \lccode`\~`\;\lowercase{\def~}{\discretionary{\hbox{\char`\;}}{\Wrappedafterbreak}{\hbox{\char`\;}}}%
            \lccode`\~`\:\lowercase{\def~}{\discretionary{\hbox{\char`\:}}{\Wrappedafterbreak}{\hbox{\char`\:}}}%
            \lccode`\~`\?\lowercase{\def~}{\discretionary{\hbox{\char`\?}}{\Wrappedafterbreak}{\hbox{\char`\?}}}%
            \lccode`\~`\!\lowercase{\def~}{\discretionary{\hbox{\char`\!}}{\Wrappedafterbreak}{\hbox{\char`\!}}}%
            \lccode`\~`\/\lowercase{\def~}{\discretionary{\hbox{\char`\/}}{\Wrappedafterbreak}{\hbox{\char`\/}}}%
            \catcode`\.\active
            \catcode`\,\active
            \catcode`\;\active
            \catcode`\:\active
            \catcode`\?\active
            \catcode`\!\active
            \catcode`\/\active
            \lccode`\~`\~
        }
    \makeatother

    \let\OriginalVerbatim=\Verbatim
    \makeatletter
    \renewcommand{\Verbatim}[1][1]{%
        %\parskip\z@skip
        \sbox\Wrappedcontinuationbox {\Wrappedcontinuationsymbol}%
        \sbox\Wrappedvisiblespacebox {\FV@SetupFont\Wrappedvisiblespace}%
        \def\FancyVerbFormatLine ##1{\hsize\linewidth
            \vtop{\raggedright\hyphenpenalty\z@\exhyphenpenalty\z@
                \doublehyphendemerits\z@\finalhyphendemerits\z@
                \strut ##1\strut}%
        }%
        % If the linebreak is at a space, the latter will be displayed as visible
        % space at end of first line, and a continuation symbol starts next line.
        % Stretch/shrink are however usually zero for typewriter font.
        \def\FV@Space {%
            \nobreak\hskip\z@ plus\fontdimen3\font minus\fontdimen4\font
            \discretionary{\copy\Wrappedvisiblespacebox}{\Wrappedafterbreak}
            {\kern\fontdimen2\font}%
        }%

        % Allow breaks at special characters using \PYG... macros.
        \Wrappedbreaksatspecials
        % Breaks at punctuation characters . , ; ? ! and / need catcode=\active
        \OriginalVerbatim[#1,codes*=\Wrappedbreaksatpunct]%
    }
    \makeatother

    % Exact colors from NB
    \definecolor{incolor}{HTML}{303F9F}
    \definecolor{outcolor}{HTML}{D84315}
    \definecolor{cellborder}{HTML}{CFCFCF}
    \definecolor{cellbackground}{HTML}{F7F7F7}

    % prompt
    \makeatletter
    \newcommand{\boxspacing}{\kern\kvtcb@left@rule\kern\kvtcb@boxsep}
    \makeatother
    \newcommand{\prompt}[4]{
        {\ttfamily\llap{{\color{#2}[#3]:\hspace{3pt}#4}}\vspace{-\baselineskip}}
    }
    

    
    % Prevent overflowing lines due to hard-to-break entities
    \sloppy
    % Setup hyperref package
    \hypersetup{
      breaklinks=true,  % so long urls are correctly broken across lines
      colorlinks=true,
      urlcolor=urlcolor,
      linkcolor=linkcolor,
      citecolor=citecolor,
      }
    % Slightly bigger margins than the latex defaults
    
    \geometry{verbose,tmargin=1in,bmargin=1in,lmargin=1in,rmargin=1in}
    
    

\begin{document}
    
    \maketitle
    
    

    
    \begin{tcolorbox}[breakable, size=fbox, boxrule=1pt, pad at break*=1mm,colback=cellbackground, colframe=cellborder]
\prompt{In}{incolor}{2}{\boxspacing}
\begin{Verbatim}[commandchars=\\\{\}]
\PY{k+kn}{import} \PY{n+nn}{numpy} \PY{k}{as} \PY{n+nn}{np}
\PY{k+kn}{import} \PY{n+nn}{matplotlib}\PY{n+nn}{.}\PY{n+nn}{pyplot} \PY{k}{as} \PY{n+nn}{plt}
\PY{k+kn}{import} \PY{n+nn}{pandas} \PY{k}{as} \PY{n+nn}{pd}
\PY{k+kn}{from} \PY{n+nn}{sklearn}\PY{n+nn}{.}\PY{n+nn}{metrics} \PY{k+kn}{import} \PY{n}{r2\PYZus{}score}
\PY{k+kn}{from} \PY{n+nn}{scipy}\PY{n+nn}{.}\PY{n+nn}{optimize} \PY{k+kn}{import} \PY{n}{curve\PYZus{}fit}
\PY{k+kn}{from} \PY{n+nn}{uncertainties} \PY{k+kn}{import} \PY{n}{ufloat}
\end{Verbatim}
\end{tcolorbox}

    \begin{tcolorbox}[breakable, size=fbox, boxrule=1pt, pad at break*=1mm,colback=cellbackground, colframe=cellborder]
\prompt{In}{incolor}{5}{\boxspacing}
\begin{Verbatim}[commandchars=\\\{\}]
\PY{c+c1}{\PYZsh{}\PYZsh{}\PYZsh{} Plotting raw data only for copper, axis = 0 \PYZsh{}\PYZsh{}\PYZsh{}}
\PY{n}{rawdata} \PY{o}{=} \PY{n}{pd}\PY{o}{.}\PY{n}{read\PYZus{}csv}\PY{p}{(}\PY{l+s+s2}{\PYZdq{}}\PY{l+s+s2}{C:}\PY{l+s+se}{\PYZbs{}\PYZbs{}}\PY{l+s+s2}{Users}\PY{l+s+se}{\PYZbs{}\PYZbs{}}\PY{l+s+s2}{chril}\PY{l+s+se}{\PYZbs{}\PYZbs{}}\PY{l+s+s2}{Documents}\PY{l+s+se}{\PYZbs{}\PYZbs{}}\PY{l+s+s2}{Python Scripts}\PY{l+s+se}{\PYZbs{}\PYZbs{}}\PY{l+s+s2}{Fluid}\PY{l+s+se}{\PYZbs{}\PYZbs{}}\PY{l+s+s2}{ENGR120\PYZhy{}Lab\PYZhy{}Report\PYZhy{}6}\PY{l+s+se}{\PYZbs{}\PYZbs{}}\PY{l+s+s2}{Python}\PY{l+s+se}{\PYZbs{}\PYZbs{}}\PY{l+s+s2}{Lab6}\PY{l+s+se}{\PYZbs{}\PYZbs{}}\PY{l+s+s2}{Copper}\PY{l+s+se}{\PYZbs{}\PYZbs{}}\PY{l+s+s2}{copper\PYZus{}fl\PYZus{}u.csv}\PY{l+s+s2}{\PYZdq{}}\PY{p}{,} \PY{n}{delimiter}\PY{o}{=}\PY{l+s+s2}{\PYZdq{}}\PY{l+s+s2}{,}\PY{l+s+s2}{\PYZdq{}}\PY{p}{,} \PY{n}{header}\PY{o}{=}\PY{k+kc}{None}\PY{p}{)}
\PY{n}{rawdata} \PY{o}{=} \PY{n}{rawdata}\PY{p}{[}\PY{n}{rawdata}\PY{o}{.}\PY{n}{iloc}\PY{p}{[}\PY{p}{:}\PY{p}{,}\PY{l+m+mi}{2}\PY{p}{]} \PY{o}{==} \PY{l+m+mi}{0}\PY{p}{]}
\PY{n}{rawdata}\PY{o}{.}\PY{n}{iloc}\PY{p}{[}\PY{p}{:}\PY{p}{,}\PY{l+m+mi}{0}\PY{p}{]} \PY{o}{=} \PY{n}{rawdata}\PY{o}{.}\PY{n}{iloc}\PY{p}{[}\PY{p}{:}\PY{p}{,}\PY{l+m+mi}{0}\PY{p}{]}\PY{o}{.}\PY{n}{replace}\PY{p}{(}\PY{l+s+s2}{\PYZdq{}}\PY{l+s+s2}{2024\PYZhy{}11\PYZhy{}12\PYZus{}}\PY{l+s+s2}{\PYZdq{}}\PY{p}{,}\PY{l+s+s2}{\PYZdq{}}\PY{l+s+s2}{\PYZdq{}}\PY{p}{,} \PY{n}{regex}\PY{o}{=}\PY{k+kc}{True}\PY{p}{)}
\PY{n}{rawdata}\PY{o}{.}\PY{n}{iloc}\PY{p}{[}\PY{p}{:}\PY{p}{,}\PY{l+m+mi}{0}\PY{p}{]} \PY{o}{=} \PY{n}{pd}\PY{o}{.}\PY{n}{to\PYZus{}datetime}\PY{p}{(}\PY{n}{rawdata}\PY{o}{.}\PY{n}{iloc}\PY{p}{[}\PY{p}{:}\PY{p}{,}\PY{l+m+mi}{0}\PY{p}{]}\PY{p}{,} \PY{n+nb}{format}\PY{o}{=}\PY{l+s+s2}{\PYZdq{}}\PY{l+s+s2}{\PYZpc{}}\PY{l+s+s2}{H:}\PY{l+s+s2}{\PYZpc{}}\PY{l+s+s2}{M:}\PY{l+s+s2}{\PYZpc{}}\PY{l+s+s2}{S.}\PY{l+s+si}{\PYZpc{}f}\PY{l+s+s2}{\PYZdq{}}\PY{p}{)}

\PY{n}{plt}\PY{o}{.}\PY{n}{plot}\PY{p}{(}\PY{n}{rawdata}\PY{o}{.}\PY{n}{iloc}\PY{p}{[}\PY{p}{:}\PY{p}{,}\PY{l+m+mi}{0}\PY{p}{]}\PY{p}{,} \PY{n}{rawdata}\PY{o}{.}\PY{n}{iloc}\PY{p}{[}\PY{p}{:}\PY{p}{,}\PY{l+m+mi}{3}\PY{p}{]}\PY{p}{,} \PY{l+s+s1}{\PYZsq{}}\PY{l+s+s1}{.}\PY{l+s+s1}{\PYZsq{}}\PY{p}{,} \PY{n}{label}\PY{o}{=}\PY{l+s+s2}{\PYZdq{}}\PY{l+s+s2}{Raw data}\PY{l+s+s2}{\PYZdq{}}\PY{p}{)}
\PY{n}{plt}\PY{o}{.}\PY{n}{xlabel}\PY{p}{(}\PY{l+s+s2}{\PYZdq{}}\PY{l+s+s2}{Time}\PY{l+s+s2}{\PYZdq{}}\PY{p}{)}
\PY{n}{plt}\PY{o}{.}\PY{n}{ylabel}\PY{p}{(}\PY{l+s+s2}{\PYZdq{}}\PY{l+s+s2}{Acceleration}\PY{l+s+s2}{\PYZdq{}}\PY{p}{)}
\PY{n}{plt}\PY{o}{.}\PY{n}{savefig}\PY{p}{(}\PY{l+s+s2}{\PYZdq{}}\PY{l+s+s2}{C:}\PY{l+s+se}{\PYZbs{}\PYZbs{}}\PY{l+s+s2}{Users}\PY{l+s+se}{\PYZbs{}\PYZbs{}}\PY{l+s+s2}{chril}\PY{l+s+se}{\PYZbs{}\PYZbs{}}\PY{l+s+s2}{Documents}\PY{l+s+se}{\PYZbs{}\PYZbs{}}\PY{l+s+s2}{Python Scripts}\PY{l+s+se}{\PYZbs{}\PYZbs{}}\PY{l+s+s2}{Fluid}\PY{l+s+se}{\PYZbs{}\PYZbs{}}\PY{l+s+s2}{ENGR120\PYZhy{}Lab\PYZhy{}Report\PYZhy{}6}\PY{l+s+se}{\PYZbs{}\PYZbs{}}\PY{l+s+s2}{Python}\PY{l+s+se}{\PYZbs{}\PYZbs{}}\PY{l+s+s2}{Lab6}\PY{l+s+se}{\PYZbs{}\PYZbs{}}\PY{l+s+s2}{Plots}\PY{l+s+se}{\PYZbs{}\PYZbs{}}\PY{l+s+s2}{rawdataCUa0.png}\PY{l+s+s2}{\PYZdq{}}\PY{p}{)}
\PY{n}{plt}\PY{o}{.}\PY{n}{close}\PY{p}{(}\PY{p}{)}
\end{Verbatim}
\end{tcolorbox}

    \begin{tcolorbox}[breakable, size=fbox, boxrule=1pt, pad at break*=1mm,colback=cellbackground, colframe=cellborder]
\prompt{In}{incolor}{8}{\boxspacing}
\begin{Verbatim}[commandchars=\\\{\}]
\PY{k}{def} \PY{n+nf}{fitfunc}\PY{p}{(}\PY{n}{x}\PY{p}{,} \PY{n}{a}\PY{p}{,} \PY{n}{b}\PY{p}{)}\PY{p}{:}
    \PY{k}{return} \PY{n}{a}\PY{o}{*}\PY{n}{x} \PY{o}{+} \PY{n}{b}
\end{Verbatim}
\end{tcolorbox}

    \begin{tcolorbox}[breakable, size=fbox, boxrule=1pt, pad at break*=1mm,colback=cellbackground, colframe=cellborder]
\prompt{In}{incolor}{6}{\boxspacing}
\begin{Verbatim}[commandchars=\\\{\}]
\PY{k}{def} \PY{n+nf}{R2Score}\PY{p}{(}\PY{n}{Material}\PY{p}{)}\PY{p}{:}
    \PY{n}{FullData} \PY{o}{=} \PY{n}{pd}\PY{o}{.}\PY{n}{DataFrame}\PY{p}{(}\PY{p}{)}
    \PY{n}{filename} \PY{o}{=} \PY{l+s+s2}{\PYZdq{}}\PY{l+s+s2}{C:}\PY{l+s+se}{\PYZbs{}\PYZbs{}}\PY{l+s+s2}{Users}\PY{l+s+se}{\PYZbs{}\PYZbs{}}\PY{l+s+s2}{chril}\PY{l+s+se}{\PYZbs{}\PYZbs{}}\PY{l+s+s2}{Documents}\PY{l+s+se}{\PYZbs{}\PYZbs{}}\PY{l+s+s2}{Python Scripts}\PY{l+s+se}{\PYZbs{}\PYZbs{}}\PY{l+s+s2}{Fluid}\PY{l+s+se}{\PYZbs{}\PYZbs{}}\PY{l+s+s2}{ENGR120\PYZhy{}Lab\PYZhy{}Report\PYZhy{}6}\PY{l+s+se}{\PYZbs{}\PYZbs{}}\PY{l+s+s2}{Python}\PY{l+s+se}{\PYZbs{}\PYZbs{}}\PY{l+s+s2}{Lab6}\PY{l+s+se}{\PYZbs{}\PYZbs{}}\PY{l+s+s2}{\PYZdq{}} \PY{o}{+} \PY{n}{Material}\PY{o}{.}\PY{n}{capitalize}\PY{p}{(}\PY{p}{)} \PY{o}{+} \PY{l+s+s2}{\PYZdq{}}\PY{l+s+se}{\PYZbs{}\PYZbs{}}\PY{l+s+s2}{\PYZdq{}} \PY{o}{+} \PY{n}{Material} \PY{o}{+} \PY{l+s+s2}{\PYZdq{}}\PY{l+s+s2}{\PYZus{}fl\PYZus{}}\PY{l+s+s2}{\PYZdq{}}
    \PY{n}{Df} \PY{o}{=} \PY{n}{pd}\PY{o}{.}\PY{n}{DataFrame}\PY{p}{(}\PY{p}{)}
    \PY{k}{for} \PY{n}{i} \PY{o+ow}{in} \PY{p}{[}\PY{l+s+s2}{\PYZdq{}}\PY{l+s+s2}{02}\PY{l+s+s2}{\PYZdq{}}\PY{p}{,} \PY{l+s+s2}{\PYZdq{}}\PY{l+s+s2}{03}\PY{l+s+s2}{\PYZdq{}}\PY{p}{,} \PY{l+s+s2}{\PYZdq{}}\PY{l+s+s2}{04}\PY{l+s+s2}{\PYZdq{}}\PY{p}{,} \PY{l+s+s2}{\PYZdq{}}\PY{l+s+s2}{06}\PY{l+s+s2}{\PYZdq{}}\PY{p}{,} \PY{l+s+s2}{\PYZdq{}}\PY{l+s+s2}{u}\PY{l+s+s2}{\PYZdq{}}\PY{p}{]}\PY{p}{:}
        \PY{n}{Data} \PY{o}{=} \PY{n}{pd}\PY{o}{.}\PY{n}{read\PYZus{}csv}\PY{p}{(}\PY{n}{filename} \PY{o}{+} \PY{n}{i} \PY{o}{+} \PY{l+s+s2}{\PYZdq{}}\PY{l+s+s2}{.csv}\PY{l+s+s2}{\PYZdq{}}\PY{p}{,} \PY{n}{delimiter} \PY{o}{=} \PY{l+s+s2}{\PYZdq{}}\PY{l+s+s2}{,}\PY{l+s+s2}{\PYZdq{}}\PY{p}{,} \PY{n}{header} \PY{o}{=} \PY{k+kc}{None}\PY{p}{)}
        \PY{n}{Df}\PY{p}{[}\PY{n}{Material} \PY{o}{+} \PY{n}{i} \PY{o}{+} \PY{l+s+s2}{\PYZdq{}}\PY{l+s+s2}{\PYZus{}time}\PY{l+s+s2}{\PYZdq{}}\PY{p}{]} \PY{o}{=} \PY{n}{Data}\PY{p}{[}\PY{l+m+mi}{0}\PY{p}{]} \PY{c+c1}{\PYZsh{}Names the columns}
        \PY{n}{Df}\PY{p}{[}\PY{n}{Material} \PY{o}{+} \PY{n}{i} \PY{o}{+} \PY{l+s+s2}{\PYZdq{}}\PY{l+s+s2}{\PYZus{}axis}\PY{l+s+s2}{\PYZdq{}}\PY{p}{]} \PY{o}{=} \PY{n}{Data}\PY{p}{[}\PY{l+m+mi}{2}\PY{p}{]}
        \PY{n}{Df}\PY{p}{[}\PY{n}{Material} \PY{o}{+} \PY{n}{i} \PY{o}{+} \PY{l+s+s2}{\PYZdq{}}\PY{l+s+s2}{\PYZus{}acceleration}\PY{l+s+s2}{\PYZdq{}}\PY{p}{]} \PY{o}{=}\PY{n}{Data}\PY{p}{[}\PY{l+m+mi}{3}\PY{p}{]}
    \PY{n}{Df} \PY{o}{=} \PY{n}{Df}\PY{o}{.}\PY{n}{dropna}\PY{p}{(}\PY{p}{)} \PY{c+c1}{\PYZsh{}Makes sure they have run for equal amount of time}

    \PY{n}{results} \PY{o}{=} \PY{n}{pd}\PY{o}{.}\PY{n}{DataFrame}\PY{p}{(}\PY{p}{)}
    \PY{k}{for} \PY{n}{axis} \PY{o+ow}{in} \PY{n+nb}{range}\PY{p}{(}\PY{l+m+mi}{3}\PY{p}{)}\PY{p}{:} \PY{c+c1}{\PYZsh{}Calculates the R2 score for each axis}
        \PY{n}{AvgAcc}\PY{p}{,} \PY{n}{stdAcc}\PY{p}{,} \PY{n}{FlowRate} \PY{o}{=} \PY{p}{[}\PY{p}{]}\PY{p}{,} \PY{p}{[}\PY{p}{]}\PY{p}{,} \PY{p}{[}\PY{p}{]}
        \PY{n}{AxisDf} \PY{o}{=} \PY{n}{Df}\PY{p}{[}\PY{n}{Df}\PY{p}{[}\PY{n}{Material} \PY{o}{+} \PY{l+s+s2}{\PYZdq{}}\PY{l+s+s2}{02\PYZus{}axis}\PY{l+s+s2}{\PYZdq{}}\PY{p}{]}  \PY{o}{==} \PY{n}{axis}\PY{p}{]} \PY{c+c1}{\PYZsh{}New temporary dataframe for each axis}

        \PY{k}{for} \PY{n}{i} \PY{o+ow}{in} \PY{p}{[}\PY{l+s+s2}{\PYZdq{}}\PY{l+s+s2}{02}\PY{l+s+s2}{\PYZdq{}}\PY{p}{,} \PY{l+s+s2}{\PYZdq{}}\PY{l+s+s2}{03}\PY{l+s+s2}{\PYZdq{}}\PY{p}{,} \PY{l+s+s2}{\PYZdq{}}\PY{l+s+s2}{04}\PY{l+s+s2}{\PYZdq{}}\PY{p}{,} \PY{l+s+s2}{\PYZdq{}}\PY{l+s+s2}{06}\PY{l+s+s2}{\PYZdq{}}\PY{p}{]}\PY{p}{:} \PY{c+c1}{\PYZsh{}Calculates the average acceleration for each flowrate}
            \PY{n}{AvgAcc}\PY{o}{.}\PY{n}{append}\PY{p}{(}\PY{n}{AxisDf}\PY{p}{[}\PY{n}{Material} \PY{o}{+} \PY{n}{i} \PY{o}{+} \PY{l+s+s2}{\PYZdq{}}\PY{l+s+s2}{\PYZus{}acceleration}\PY{l+s+s2}{\PYZdq{}}\PY{p}{]}\PY{o}{.}\PY{n}{mean}\PY{p}{(}\PY{p}{)}\PY{p}{)} 
            \PY{n}{stdAcc}\PY{o}{.}\PY{n}{append}\PY{p}{(}\PY{n}{AxisDf}\PY{p}{[}\PY{n}{Material} \PY{o}{+} \PY{n}{i} \PY{o}{+} \PY{l+s+s2}{\PYZdq{}}\PY{l+s+s2}{\PYZus{}acceleration}\PY{l+s+s2}{\PYZdq{}}\PY{p}{]}\PY{o}{.}\PY{n}{std}\PY{p}{(}\PY{p}{)}\PY{o}{/}\PY{n}{np}\PY{o}{.}\PY{n}{sqrt}\PY{p}{(}\PY{n+nb}{len}\PY{p}{(}\PY{n}{AxisDf}\PY{p}{[}\PY{n}{Material} \PY{o}{+} \PY{n}{i} \PY{o}{+} \PY{l+s+s2}{\PYZdq{}}\PY{l+s+s2}{\PYZus{}acceleration}\PY{l+s+s2}{\PYZdq{}}\PY{p}{]}\PY{p}{)}\PY{p}{)}\PY{p}{)}
            \PY{n}{FlowRate}\PY{o}{.}\PY{n}{append}\PY{p}{(}\PY{n+nb}{int}\PY{p}{(}\PY{n}{i}\PY{p}{)}\PY{p}{)}

        \PY{n}{par}\PY{p}{,} \PY{n}{cov} \PY{o}{=} \PY{n}{curve\PYZus{}fit}\PY{p}{(}\PY{n}{fitfunc}\PY{p}{,} \PY{n}{FlowRate}\PY{p}{,} \PY{n}{AvgAcc}\PY{p}{,} \PY{n}{sigma} \PY{o}{=} \PY{n}{stdAcc}\PY{p}{,} \PY{n}{absolute\PYZus{}sigma}\PY{o}{=}\PY{k+kc}{True}\PY{p}{)} \PY{c+c1}{\PYZsh{}Fits the data to a linear function}

        \PY{n}{unkownAcc} \PY{o}{=} \PY{n}{AxisDf}\PY{p}{[}\PY{n}{Material} \PY{o}{+} \PY{l+s+s2}{\PYZdq{}}\PY{l+s+s2}{u\PYZus{}acceleration}\PY{l+s+s2}{\PYZdq{}}\PY{p}{]}\PY{o}{.}\PY{n}{mean}\PY{p}{(}\PY{p}{)} \PY{c+c1}{\PYZsh{}Calculates the unknown flowrate}
        
        \PY{n+nb}{dict} \PY{o}{=} \PY{p}{\PYZob{}}
            \PY{l+s+s2}{\PYZdq{}}\PY{l+s+s2}{Material}\PY{l+s+s2}{\PYZdq{}}\PY{p}{:} \PY{p}{[}\PY{n}{Material}\PY{p}{]}\PY{p}{,} 
            \PY{l+s+s2}{\PYZdq{}}\PY{l+s+s2}{Axis}\PY{l+s+s2}{\PYZdq{}}\PY{p}{:} \PY{p}{[}\PY{n}{axis}\PY{p}{]}\PY{p}{,}  
            \PY{l+s+s2}{\PYZdq{}}\PY{l+s+s2}{R2\PYZus{}score}\PY{l+s+s2}{\PYZdq{}}\PY{p}{:} \PY{p}{[}\PY{n}{r2\PYZus{}score}\PY{p}{(}\PY{n}{AvgAcc}\PY{p}{,} \PY{n}{fitfunc}\PY{p}{(}\PY{n}{np}\PY{o}{.}\PY{n}{asarray}\PY{p}{(}\PY{n}{FlowRate}\PY{p}{)}\PY{p}{,} \PY{o}{*}\PY{n}{par}\PY{p}{)}\PY{p}{)}\PY{p}{]}\PY{p}{,} 
            \PY{l+s+s2}{\PYZdq{}}\PY{l+s+s2}{Unknown flowrate}\PY{l+s+s2}{\PYZdq{}}\PY{p}{:} \PY{p}{[}\PY{p}{(}\PY{n}{unkownAcc} \PY{o}{\PYZhy{}} \PY{n}{par}\PY{p}{[}\PY{l+m+mi}{1}\PY{p}{]}\PY{p}{)}\PY{o}{/}\PY{n}{par}\PY{p}{[}\PY{l+m+mi}{0}\PY{p}{]} \PY{p}{]}\PY{p}{,}
            \PY{l+s+s2}{\PYZdq{}}\PY{l+s+s2}{Flowrate errror}\PY{l+s+s2}{\PYZdq{}}\PY{p}{:} \PY{p}{[}\PY{n}{np}\PY{o}{.}\PY{n}{sqrt}\PY{p}{(}\PY{p}{(}\PY{l+m+mi}{1}\PY{o}{/}\PY{n}{par}\PY{p}{[}\PY{l+m+mi}{0}\PY{p}{]}\PY{p}{)}\PY{o}{*}\PY{o}{*}\PY{l+m+mi}{2}\PY{o}{*} \PY{p}{(}\PY{n}{np}\PY{o}{.}\PY{n}{std}\PY{p}{(}\PY{n}{AxisDf}\PY{p}{[}\PY{n}{Material} \PY{o}{+} \PY{l+s+s2}{\PYZdq{}}\PY{l+s+s2}{u\PYZus{}acceleration}\PY{l+s+s2}{\PYZdq{}}\PY{p}{]}\PY{p}{)}\PY{o}{/}\PY{n+nb}{len}\PY{p}{(}\PY{n}{AxisDf}\PY{p}{[}\PY{n}{Material} \PY{o}{+} \PY{l+s+s2}{\PYZdq{}}\PY{l+s+s2}{u\PYZus{}acceleration}\PY{l+s+s2}{\PYZdq{}}\PY{p}{]}\PY{p}{)}\PY{p}{)}\PY{o}{*}\PY{o}{*}\PY{l+m+mi}{2}
                                        \PY{o}{+} \PY{p}{(}\PY{l+m+mi}{1}\PY{o}{/}\PY{n}{par}\PY{p}{[}\PY{l+m+mi}{0}\PY{p}{]}\PY{p}{)}\PY{o}{*}\PY{o}{*}\PY{l+m+mi}{2}\PY{o}{*}\PY{n}{cov}\PY{p}{[}\PY{l+m+mi}{1}\PY{p}{,}\PY{l+m+mi}{1}\PY{p}{]} 
                                        \PY{o}{+} \PY{p}{(}\PY{p}{(}\PY{n}{unkownAcc} \PY{o}{\PYZhy{}} \PY{n}{par}\PY{p}{[}\PY{l+m+mi}{1}\PY{p}{]}\PY{p}{)}\PY{o}{/}\PY{p}{(}\PY{n}{par}\PY{p}{[}\PY{l+m+mi}{0}\PY{p}{]}\PY{o}{*}\PY{o}{*}\PY{l+m+mi}{2}\PY{p}{)}\PY{p}{)}\PY{o}{*}\PY{o}{*}\PY{l+m+mi}{2}\PY{o}{*}\PY{n}{cov}\PY{p}{[}\PY{l+m+mi}{0}\PY{p}{,}\PY{l+m+mi}{0}\PY{p}{]}\PY{p}{)}\PY{p}{]}\PY{p}{,}
            \PY{p}{\PYZcb{}} \PY{c+c1}{\PYZsh{}Creates a dictionary with the results}
        \PY{n}{results} \PY{o}{=} \PY{n}{pd}\PY{o}{.}\PY{n}{DataFrame}\PY{p}{(}\PY{n+nb}{dict}\PY{p}{)} \PY{c+c1}{\PYZsh{}Creates a dataframe with the results}
        \PY{c+c1}{\PYZsh{}write to csv}
        \PY{n}{results}\PY{o}{.}\PY{n}{to\PYZus{}csv}\PY{p}{(}\PY{l+s+s2}{\PYZdq{}}\PY{l+s+s2}{C:}\PY{l+s+se}{\PYZbs{}\PYZbs{}}\PY{l+s+s2}{Users}\PY{l+s+se}{\PYZbs{}\PYZbs{}}\PY{l+s+s2}{chril}\PY{l+s+se}{\PYZbs{}\PYZbs{}}\PY{l+s+s2}{Documents}\PY{l+s+se}{\PYZbs{}\PYZbs{}}\PY{l+s+s2}{Python Scripts}\PY{l+s+se}{\PYZbs{}\PYZbs{}}\PY{l+s+s2}{Fluid}\PY{l+s+se}{\PYZbs{}\PYZbs{}}\PY{l+s+s2}{ENGR120\PYZhy{}Lab\PYZhy{}Report\PYZhy{}6}\PY{l+s+se}{\PYZbs{}\PYZbs{}}\PY{l+s+s2}{Python}\PY{l+s+se}{\PYZbs{}\PYZbs{}}\PY{l+s+s2}{Lab6}\PY{l+s+se}{\PYZbs{}\PYZbs{}}\PY{l+s+s2}{Results}\PY{l+s+se}{\PYZbs{}\PYZbs{}}\PY{l+s+s2}{\PYZdq{}} \PY{o}{+} \PY{n}{Material} \PY{o}{+} \PY{n+nb}{str}\PY{p}{(}\PY{n}{axis}\PY{p}{)}\PY{o}{+} \PY{l+s+s2}{\PYZdq{}}\PY{l+s+s2}{\PYZus{}R2\PYZus{}score.csv}\PY{l+s+s2}{\PYZdq{}}\PY{p}{,} \PY{n}{header} \PY{o}{=} \PY{k+kc}{True}\PY{p}{)}
        
        \PY{k}{if} \PY{n}{axis} \PY{o}{==} \PY{l+m+mi}{0}\PY{p}{:} \PY{c+c1}{\PYZsh{}Plots the data for all materials, only for axis = 0}
            \PY{n}{a} \PY{o}{=} \PY{n}{ufloat}\PY{p}{(}\PY{n}{par}\PY{p}{[}\PY{l+m+mi}{0}\PY{p}{]}\PY{p}{,} \PY{n}{np}\PY{o}{.}\PY{n}{sqrt}\PY{p}{(}\PY{n}{cov}\PY{p}{[}\PY{l+m+mi}{0}\PY{p}{,}\PY{l+m+mi}{0}\PY{p}{]}\PY{p}{)}\PY{p}{)} \PY{c+c1}{\PYZsh{}Calculates the parameters and uncertainties}
            \PY{n}{b} \PY{o}{=} \PY{n}{ufloat}\PY{p}{(}\PY{n}{par}\PY{p}{[}\PY{l+m+mi}{1}\PY{p}{]}\PY{p}{,} \PY{n}{np}\PY{o}{.}\PY{n}{sqrt}\PY{p}{(}\PY{n}{cov}\PY{p}{[}\PY{l+m+mi}{1}\PY{p}{,}\PY{l+m+mi}{1}\PY{p}{]}\PY{p}{)}\PY{p}{)}
            \PY{n}{plt}\PY{o}{.}\PY{n}{errorbar}\PY{p}{(}\PY{n}{FlowRate}\PY{p}{,} \PY{n}{AvgAcc}\PY{p}{,} \PY{n}{yerr} \PY{o}{=} \PY{n}{stdAcc}\PY{p}{,} \PY{n}{fmt} \PY{o}{=} \PY{l+s+s2}{\PYZdq{}}\PY{l+s+s2}{o}\PY{l+s+s2}{\PYZdq{}}\PY{p}{,} \PY{n}{label} \PY{o}{=} \PY{l+s+s2}{\PYZdq{}}\PY{l+s+s2}{Data}\PY{l+s+s2}{\PYZdq{}}\PY{p}{)}
            \PY{n}{plt}\PY{o}{.}\PY{n}{plot}\PY{p}{(}
                \PY{n}{FlowRate}\PY{p}{,} 
                \PY{n}{fitfunc}\PY{p}{(}\PY{n}{np}\PY{o}{.}\PY{n}{asarray}\PY{p}{(}\PY{n}{FlowRate}\PY{p}{)}\PY{p}{,} \PY{o}{*}\PY{n}{par}\PY{p}{)}\PY{p}{,} 
                \PY{n}{label} \PY{o}{=} \PY{l+s+sa}{f}\PY{l+s+s2}{\PYZdq{}}\PY{l+s+s2}{Fit, \PYZdl{}a = }\PY{l+s+si}{\PYZob{}}\PY{n}{a}\PY{l+s+si}{:}\PY{l+s+s2}{L}\PY{l+s+si}{\PYZcb{}}\PY{l+s+s2}{, b = }\PY{l+s+si}{\PYZob{}}\PY{n}{b}\PY{l+s+si}{:}\PY{l+s+s2}{L}\PY{l+s+si}{\PYZcb{}}\PY{l+s+s2}{\PYZdl{}}\PY{l+s+s2}{\PYZdq{}}
                \PY{p}{)}
            \PY{n}{plt}\PY{o}{.}\PY{n}{xlabel}\PY{p}{(}\PY{l+s+s2}{\PYZdq{}}\PY{l+s+s2}{Flowrate}\PY{l+s+s2}{\PYZdq{}}\PY{p}{)}
            \PY{n}{plt}\PY{o}{.}\PY{n}{ylabel}\PY{p}{(}\PY{l+s+s2}{\PYZdq{}}\PY{l+s+s2}{Acceleration}\PY{l+s+s2}{\PYZdq{}}\PY{p}{)}
            \PY{n}{plt}\PY{o}{.}\PY{n}{legend}\PY{p}{(}\PY{p}{)}
            \PY{n}{plt}\PY{o}{.}\PY{n}{savefig}\PY{p}{(}\PY{l+s+s2}{\PYZdq{}}\PY{l+s+s2}{C:}\PY{l+s+se}{\PYZbs{}\PYZbs{}}\PY{l+s+s2}{Users}\PY{l+s+se}{\PYZbs{}\PYZbs{}}\PY{l+s+s2}{chril}\PY{l+s+se}{\PYZbs{}\PYZbs{}}\PY{l+s+s2}{Documents}\PY{l+s+se}{\PYZbs{}\PYZbs{}}\PY{l+s+s2}{Python Scripts}\PY{l+s+se}{\PYZbs{}\PYZbs{}}\PY{l+s+s2}{Fluid}\PY{l+s+se}{\PYZbs{}\PYZbs{}}\PY{l+s+s2}{ENGR120\PYZhy{}Lab\PYZhy{}Report\PYZhy{}6}\PY{l+s+se}{\PYZbs{}\PYZbs{}}\PY{l+s+s2}{Python}\PY{l+s+se}{\PYZbs{}\PYZbs{}}\PY{l+s+s2}{Lab6}\PY{l+s+se}{\PYZbs{}\PYZbs{}}\PY{l+s+s2}{Plots}\PY{l+s+se}{\PYZbs{}\PYZbs{}}\PY{l+s+s2}{\PYZdq{}} \PY{o}{+} \PY{n}{Material} \PY{o}{+} \PY{l+s+s2}{\PYZdq{}}\PY{l+s+s2}{\PYZus{}axis}\PY{l+s+s2}{\PYZdq{}} \PY{o}{+} \PY{n+nb}{str}\PY{p}{(}\PY{n}{axis}\PY{p}{)} \PY{o}{+} \PY{l+s+s2}{\PYZdq{}}\PY{l+s+s2}{.png}\PY{l+s+s2}{\PYZdq{}}\PY{p}{)}
            \PY{n}{plt}\PY{o}{.}\PY{n}{close}\PY{p}{(}\PY{p}{)}
\end{Verbatim}
\end{tcolorbox}

    \begin{tcolorbox}[breakable, size=fbox, boxrule=1pt, pad at break*=1mm,colback=cellbackground, colframe=cellborder]
\prompt{In}{incolor}{9}{\boxspacing}
\begin{Verbatim}[commandchars=\\\{\}]
\PY{k}{for} \PY{n}{mat} \PY{o+ow}{in} \PY{p}{[}\PY{l+s+s2}{\PYZdq{}}\PY{l+s+s2}{copper}\PY{l+s+s2}{\PYZdq{}}\PY{p}{,} \PY{l+s+s2}{\PYZdq{}}\PY{l+s+s2}{galvanized}\PY{l+s+s2}{\PYZdq{}}\PY{p}{,} \PY{l+s+s2}{\PYZdq{}}\PY{l+s+s2}{pvc}\PY{l+s+s2}{\PYZdq{}}\PY{p}{]}\PY{p}{:}
    \PY{n}{R2Score}\PY{p}{(}\PY{n}{mat}\PY{p}{)}
\end{Verbatim}
\end{tcolorbox}

    \begin{tcolorbox}[breakable, size=fbox, boxrule=1pt, pad at break*=1mm,colback=cellbackground, colframe=cellborder]
\prompt{In}{incolor}{ }{\boxspacing}
\begin{Verbatim}[commandchars=\\\{\}]
\PY{n}{Fulldata} \PY{o}{=} \PY{n}{pd}\PY{o}{.}\PY{n}{DataFrame}\PY{p}{(}\PY{p}{)}
\PY{k}{for} \PY{n}{mat} \PY{o+ow}{in} \PY{p}{[}\PY{l+s+s2}{\PYZdq{}}\PY{l+s+s2}{copper}\PY{l+s+s2}{\PYZdq{}}\PY{p}{,} \PY{l+s+s2}{\PYZdq{}}\PY{l+s+s2}{galvanized}\PY{l+s+s2}{\PYZdq{}}\PY{p}{,} \PY{l+s+s2}{\PYZdq{}}\PY{l+s+s2}{pvc}\PY{l+s+s2}{\PYZdq{}}\PY{p}{]}\PY{p}{:}
    \PY{k}{for} \PY{n}{axis} \PY{o+ow}{in} \PY{n+nb}{range}\PY{p}{(}\PY{l+m+mi}{3}\PY{p}{)}\PY{p}{:}
        \PY{n}{data} \PY{o}{=} \PY{n}{pd}\PY{o}{.}\PY{n}{read\PYZus{}csv}\PY{p}{(}\PY{l+s+s2}{\PYZdq{}}\PY{l+s+s2}{C:}\PY{l+s+se}{\PYZbs{}\PYZbs{}}\PY{l+s+s2}{Users}\PY{l+s+se}{\PYZbs{}\PYZbs{}}\PY{l+s+s2}{chril}\PY{l+s+se}{\PYZbs{}\PYZbs{}}\PY{l+s+s2}{Documents}\PY{l+s+se}{\PYZbs{}\PYZbs{}}\PY{l+s+s2}{Python Scripts}\PY{l+s+se}{\PYZbs{}\PYZbs{}}\PY{l+s+s2}{Fluid}\PY{l+s+se}{\PYZbs{}\PYZbs{}}\PY{l+s+s2}{ENGR120\PYZhy{}Lab\PYZhy{}Report\PYZhy{}6}\PY{l+s+se}{\PYZbs{}\PYZbs{}}\PY{l+s+s2}{Python}\PY{l+s+se}{\PYZbs{}\PYZbs{}}\PY{l+s+s2}{Lab6}\PY{l+s+se}{\PYZbs{}\PYZbs{}}\PY{l+s+s2}{Results}\PY{l+s+se}{\PYZbs{}\PYZbs{}}\PY{l+s+s2}{\PYZdq{}} \PY{o}{+} \PY{n}{mat} \PY{o}{+} \PY{n+nb}{str}\PY{p}{(}\PY{n}{axis}\PY{p}{)} \PY{o}{+} \PY{l+s+s2}{\PYZdq{}}\PY{l+s+s2}{\PYZus{}R2\PYZus{}score.csv}\PY{l+s+s2}{\PYZdq{}}\PY{p}{,} \PY{n}{delimiter} \PY{o}{=} \PY{l+s+s2}{\PYZdq{}}\PY{l+s+s2}{,}\PY{l+s+s2}{\PYZdq{}}\PY{p}{,} \PY{n}{header} \PY{o}{=} \PY{l+m+mi}{0}\PY{p}{)}
        \PY{n+nb}{print}\PY{p}{(}\PY{n}{data}\PY{p}{[}\PY{l+s+s2}{\PYZdq{}}\PY{l+s+s2}{Material}\PY{l+s+s2}{\PYZdq{}}\PY{p}{]}\PY{o}{.}\PY{n}{values}\PY{p}{[}\PY{l+m+mi}{0}\PY{p}{]}\PY{p}{,} 
              \PY{n+nb}{round}\PY{p}{(}\PY{n}{data}\PY{p}{[}\PY{l+s+s2}{\PYZdq{}}\PY{l+s+s2}{Axis}\PY{l+s+s2}{\PYZdq{}}\PY{p}{]}\PY{o}{.}\PY{n}{values}\PY{p}{[}\PY{l+m+mi}{0}\PY{p}{]}\PY{p}{,} \PY{l+m+mi}{2}\PY{p}{)}\PY{p}{,} 
              \PY{n+nb}{round}\PY{p}{(}\PY{n}{data}\PY{p}{[}\PY{l+s+s2}{\PYZdq{}}\PY{l+s+s2}{R2\PYZus{}score}\PY{l+s+s2}{\PYZdq{}}\PY{p}{]}\PY{o}{.}\PY{n}{values}\PY{p}{[}\PY{l+m+mi}{0}\PY{p}{]}\PY{p}{,} \PY{l+m+mi}{2}\PY{p}{)}\PY{p}{,} 
                \PY{l+s+s2}{\PYZdq{}}\PY{l+s+s2}{\PYZdl{}}\PY{l+s+s2}{\PYZob{}}\PY{l+s+s2}{:L\PYZcb{}\PYZdl{}}\PY{l+s+s2}{\PYZdq{}}\PY{o}{.}\PY{n}{format}\PY{p}{(}\PY{n}{ufloat}\PY{p}{(}\PY{n}{data}\PY{p}{[}\PY{l+s+s2}{\PYZdq{}}\PY{l+s+s2}{Unknown flowrate}\PY{l+s+s2}{\PYZdq{}}\PY{p}{]}\PY{o}{.}\PY{n}{values}\PY{p}{[}\PY{l+m+mi}{0}\PY{p}{]}\PY{p}{,} \PY{n}{data}\PY{p}{[}\PY{l+s+s2}{\PYZdq{}}\PY{l+s+s2}{Flowrate errror}\PY{l+s+s2}{\PYZdq{}}\PY{p}{]}\PY{o}{.}\PY{n}{values}\PY{p}{[}\PY{l+m+mi}{0}\PY{p}{]}\PY{p}{)}\PY{p}{)}\PY{o}{.}\PY{n}{replace}\PY{p}{(}\PY{l+s+s2}{\PYZdq{}}\PY{l+s+s2}{ }\PY{l+s+s2}{\PYZdq{}}\PY{p}{,} \PY{l+s+s2}{\PYZdq{}}\PY{l+s+s2}{\PYZdq{}}\PY{p}{)}
        \PY{p}{)}
\end{Verbatim}
\end{tcolorbox}

    \begin{Verbatim}[commandchars=\\\{\}]
copper 0 0.73 \$5.3\textbackslash{}pm3.3\$
copper 1 0.38 \$6.0\textbackslash{}pm1.8\$
copper 2 0.5 \$0\textbackslash{}pm8\$
galvanized 0 0.64 \$4.3\textbackslash{}pm2.9\$
galvanized 1 0.27 \$3\textbackslash{}pm4\$
galvanized 2 0.39 \$10\textbackslash{}pm9\$
pvc 0 0.9 \$6.2\textbackslash{}pm1.0\$
pvc 1 0.07 \$-7\textbackslash{}pm11\$
pvc 2 -0.37 \$1\textbackslash{}pm5\$
    \end{Verbatim}

    \begin{tcolorbox}[breakable, size=fbox, boxrule=1pt, pad at break*=1mm,colback=cellbackground, colframe=cellborder]
\prompt{In}{incolor}{ }{\boxspacing}
\begin{Verbatim}[commandchars=\\\{\}]
\PY{o}{!}jupyter nbconvert \PYZhy{}\PYZhy{}to markdown Lab6.ipynb
\end{Verbatim}
\end{tcolorbox}


    % Add a bibliography block to the postdoc
    
    
    
\end{document}


}


\nocite{*}
\section*{\underline{Citations}}
\bibliography{aapmsamp}% Produces the bibliography via BibTeX.


\end{document}
%
% ****** End of file aapmsamp.tex ******
